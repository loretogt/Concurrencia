\documentclass{article}

\usepackage{svn-multi}
\svnid{$Id: ejercicios-cortos.tex 1638 2017-02-07 12:39:10Z xmc $}

\usepackage[a4paper]{geometry}

\setlength{\parskip}{1.25ex}

\usepackage[utf8]{inputenc}

\usepackage{hyperref}

\usepackage{xcolor}

\usepackage{../../share/tex/cc}
\usepackage{../../share/tex/notacion-ctads}

% \usepackage{graphics} % Or
% \usepackage{graphicx} %% and then \includegraphics[scale=0.5,...]{filename}

\usepackage[spanish]{babel}
%\usepackage[spanish,english]{babel}

% \usepackage{url}

% \usepackage{fancybox}
\usepackage{fancyhdr}

% \usepackage{courier}
% \usepackage{times}
% \usepackage{fourier}
\usepackage{charter}

\title{
  Cuadernillo de ejercicios cortos\\
  \asignatura.
  %\convocatoria\
  Curso~\curso
}

\author{Ángel Herranz \and Julio Mariño}
\date{Versión \svnrev.\\Última actualización: \svndate.}

\begin{document}
\pagestyle{fancy}
\lhead{Hoja de ejercicios cortos (r\svnrev)}
\chead{\asignatura}
\rhead{\convocatoria\ \curso}
\lfoot{}
\cfoot{\thepage}
\rfoot{}

\maketitle

\noindent
Este documento contiene los enunciados de los ejercicios que los
alumnos deben entregar cada semana en la asignatura de Concurrencia.
La entrega se realizará a través de la página
\url{http://lml.ls.fi.upm.es/entrega}. Es obligatorio entregar fichero
Java con codificación de caracteres UTF-8 (configura para ello tu
editor o entorno de desarrollo favorito).

\noindent\textbf{Nota:} cuidado al copiar y pegar código de este PDF,
puede que el resultado no sea el esperado en el editor.

\tableofcontents

\lstset{
  style=code,
  language=Java,
  frame=single,
  xleftmargin=2mm,
  xrightmargin=2mm,
  backgroundcolor=\color{lightgray},
}

\clearpage
\section{Creación de threads en Java}
Con lo visto en clase y la documentación sobre concurrencia en los
tutoriales de Java
(\url{http://docs.oracle.com/javase/tutorial/essential/concurrency/},
se pide escribir un programa concurrente en Java que arranque $N$
threads y termine cuando los $N$ threads terminen. Todos los threads
deben realizar el mismo trabajo: imprimir una línea que los
identifique y distinga (no se permite el uso de
\lstinline{Thread.currentThread()} ni los métodos \lstinline{getId()}
o \lstinline{toString()} o \lstinline{getName()} de la clase
\lstinline{Thread}), dormir durante $T$ milisegundos y terminar
imprimiendo una línea que los identifique y distinga. El \emph{thread}
principal, además de poner en marcha todos los procesos, debe imprimir
una línea avisando de que todos los threads han terminado una vez que
lo hayan hecho.

Es un ejercicio muy sencillo que debe servir para jugar con el
concepto de proceso intentando distinguir lo que cada proceso hace y
el momento en el que lo hace. Además, se podrá observar cómo cada
ejecución lleva a resultados diferentes. Se sugiere jugar con los
valores $N$ y $T$ e incluso hacer que $T$ sea distinto para cada
proceso.

\subsection*{Material a entregar}
El fichero fuente a entregar debe llamarse \url{CC_01_Threads.java}

\clearpage
\section{Provocar una condición de carrera}
Este ejercicio consiste en escribir un programa concurrente en el que
múltiples threads compartan y modifiquen una variable de tipo int de
forma que el resultado final de la variable una vez que los threads
terminan no sea el valor esperado. Seamos más concretos. Tendremos dos
tipos de procesos, decrementadores e incrementadores que realizan $N$
decrementos e incrementos, respectivamente, sobre una misma variable
(\lstinline{n}) de tipo \lstinline{int} inicializada a 0. El programa
concurrente pondrá en marcha $M$ procesos de cada tipo y una vez que
todos los threads han terminado imprimirá el valor de la variable
compartida.

El valor final de la variable debería ser 0 ya que se habrán producido
$M\times N$ decrementos (\lstinline{n--}) y $M\times N$ incrementos
(\lstinline{n++}), sin embargo, si dos operaciones (tanto de
decremento como de incremento) se realizan a la vez el resultado puede
no ser el esperado (por ejemplo, dos incrementos podrían terminar por no
incrementar la variable en 2).

El alumno no debería realizar la entrega hasta que no vea que el valor
final de la variable puede ser distinto de 0 (aunque esto no garantiza
que haya una condición de carrera).

\subsection*{Material a entregar}
El fichero fuente a entregar debe llamarse \url{CC_02_Carrera.java}.

\clearpage

\section{Garantizar exclusión mutua con espera activa}
Este ejercicio consiste en evitar la condición de carrera que se
produjo en el ejercicio anterior. Para ello supondremos la existencia
de \textbf{sólo dos procesos}, que simultáneamente ejecutan sendos
bucles de $N$ pasos incrementando y decrementando, respectivamente, en
cada paso una variable compartida (la operación de incremento y la de
decremento sobre esa misma variable compartida son secciones
críticas). El objetivo es evitar que mientras un proceso modifica la
variable el otro haga lo mismo (propiedad que se denomina exclusión
mutua: no puede haber dos procesos modificando simultáneamente esa
variable) y el objetivo es hacerlo utilizando sólo nuevas variables y
``espera activa'' (en otras palabras, está prohibido utilizar métodos
synchronized, semáforos o cualquier otro mecanismo de concurrencia).

\subsection*{Material a entregar}
El fichero fuente a entregar debe llamarse \url{CC_03_MutexEA.java}.

\subsection*{Material de apoyo}
Se ofrece un \textbf{intento
fallido} para asegurar la exclusión mutua en la ejecución de las
secciones críticas en el fichero \url{CC_03_MutexEA.todo.java}:
\lstinputlisting{CC_03_MutexEA.todo.java}

\clearpage
\section{Garantizar exclusión mutua con semáforos}
Este ejercicio, al igual que el anterior, consiste en evitar una
condición de carrera. En esta ocasión tenemos el mismo número de
procesos incrementadores que decrementadores que incrementan y
decrementan, respectivamente, en un mismo número de pasos una variable
compartida. El objetivo es asegurar la exclusión mutua en la ejecución
de los incrementos y decrementos de la variable y el objetivo es
hacerlo utilizando exclusivamente un semáforo de la clase
\lstinline{es.upm.babel.cclib.Semaphore} (está prohibido utilizar cualquier otro
mecanismo de concurrencia). La librería de concurrencia \url{cclib.jar}
puede descargarse de la página web de la asignatura.

\subsection*{Material a entregar}
El fichero fuente a entregar debe llamarse \url{CC_04_MutexSem.java}.

\subsection*{Material de apoyo}
Se ofrece el fichero \url{CC_04_MutexSem.todo.java} con un esqueleto
que debe respetarse y que muestra una condición de carrera:
\lstinputlisting{CC_04_MutexSem.todo.java}

\clearpage
\section{Almacén de un dato con semáforos}
En este caso nos enfrentamos a un típico programa de concurrencia:
productores-consumidores. Existen procesos de dos tipos diferentes:
\begin{itemize}
\item Productores: su hilo de ejecución consiste, repetidamente, en
  crear un producto (ver la clase \lstinline{es.upm.babel.cclib.Producto}) y
  hacerlo llegar a uno de los consumidores.
\item Consumidores: su hilo de ejecución consiste, repetidamente,
  recoger productos producidos por los productores y consumirlos.
\end{itemize}
Las clases que implementan ambos threads forman parte de la librería
CCLib:
\newline
\lstinline{es.upm.babel.cclib.Productor} y
\lstinline{es.upm.babel.cclib.Consumidor}.

La comunicación entre productores y consumidores se realizará a través
de un ``almacén'' compartido por todos los procesos. Dicho objeto
respetará la interfaz \lstinline{es.upm.babel.cclib.Almacen}:
\lstinputlisting{../../../lib/cclib/es/upm/babel/cclib/Almacen.java}

Se pide implementar sólo con semáforos una clase que siga dicha
interfaz. Sólo puede haber almacenado como máximo un producto, si un
proceso quiere almacenar debe esperar hasta que no haya un producto y
si un proceso quiera extraer espere hasta que haya un
producto. Téngase en cuenta además los posible problemas de no
asegurar la exclusión mutua en el acceso a los atributos
compartidos.

\subsection*{Material a entregar}
El fichero fuente a entregar debe llamarse \url{Almacen1.java}.

\subsection*{Material de apoyo}
Se ofrece un esqueleto de código que el alumno debe completar en el fichero
\url{Almacen1.todo.java}:
\lstinputlisting{Almacen1.todo.java}

El programa principal \url{CC_05_P1CSem.java} es el siguiente:
\lstinputlisting{CC_05_P1CSem.java}

Para valorar si el problema está bien resuelto, os recordamos que el
objetivo es asegurar
\begin{enumerate}
\item que todos los productos producidos acaban por ser consumidos,
\item que no se consume un producto dos veces y
\item que no se consume ningún producto no válido (null, por ejemplo).
\end{enumerate}

\noindent\textbf{Recomendación:} jugar con los valores de número de
productores y número de consumidores y observar con atención las
trazas del programa.

\clearpage
\section{Almacén de varios datos con semáforos}
Este ejercicio es una variación del problema anterior. En esta
ocasión, el almacén a implementar tiene una capacidad de hasta $N$
productos, lo que permite a los productores seguir trabajando aunque
los consumidores se vuelvan, momentáneamente, lentos.

\subsection*{Material a entregar}
El fichero fuente a entregar debe llamarse \url{AlmacenN.java}.

\subsection*{Material de apoyo}
Se ofrece un esqueleto de código que el alumno debe completar en el fichero
\url{AlmacenN.todo.java}:
\lstinputlisting{AlmacenN.todo.java}

y el programa principal \url{CC_06_PNSem.java} es el siguiente:
\lstinputlisting{CC_06_PNCSem.java}

\clearpage
\section{Especificación de un recurso compartido}
El ejercicio consiste en \textbf{elaborar la especificación formal del
  recurso compartido ControlAccesoPuente}. El recurso compartido forma
parte de un programa concurrente que gestiona los accesos y salidas de
coches de un puente de un solo carril. El puente tiene dos entradas y
dos salidas.  Los coches que entran por la entrada sur salen por la
salida norte y viceversa. En cada entrada existe un detector y una
barrera. En cada salida existe un detector. El sistema de detección y
barreras tienen el interfaz de control \url{Puente.java}\footnote{Los
  comentarios tipo TODO no significan que el alumno tenga que
  implementar nada, sólo son apuntes para disponer eventualmente de
  una simulación.}:
\lstinputlisting{Puente.java}

El objetivo del programa concurrente es controlar la entrada y salida
de vehículos de tal forma que jamás haya en el puente dos coches que
pretenda dirigirse en sentido contrario.  Se ha decidido un diseño en
el que existe un proceso por cada entrada y un proceso por cada
salida. Los procesos comparten un recurso compartido que les sirve
para comunicarse.

A continuación se muestra el interfaz del recurso
compartido que hay que especificar \linebreak (\url{ControlAccesoPuente.java}):
\lstinputlisting{ControlAccesoPuente.java}
y el programa concurrente \url{CC_07_Puente.java}:
\lstinputlisting{CC_07_Puente.java}

\clearpage
\section{Multibuffer con métodos synchronized}
El \emph{MultiBuffer} es una variación del problema del búffer
compartido en el que productores y consumidores pueden insertar o
extraer secuencias de elementos de longitud arbitraria, lo cual lo
convierte en un ejemplo más realista. A diferencia de versiones más
sencillas, este es un ejercicio de programación difícil si sólo se
dispone de mecanismos de sincronización de bajo nivel
(p.ej.~semáforos). 

Por ello, os pedimos que lo implementéis en Java traduciendo la
siguiente especificación de recurso a una clase usando métodos
\emph{synchronized} y el mecanismo \texttt{wait()/notifyAll()}. 

\begin{ctadsol}
  \nombrectad{MultiBuffer}
  \\
  \operaciones
  \operacion{Poner}{Tipo\_Secuencia[e]}
  \operacion{Tomar}{\nat[e] \por Tipo\_Secuencia[s]}
  \\
  \semdom
  \tipo{MultiBuffer = \secn{Tipo\_Dato} \\
        Tipo\_Secuencia =  MultiBuffer}
  \invariante{$\lng{\self} \leq MAX$}
  \donde{$MAX = ...$}
  \inicial{\self}{$\self = \secv$}
  \\
  \pre{$n \leq \lfloor MAX / 2 \rfloor$}
  \cprenl{Hay suficientes elementos en el multibuffer}
  \cpre{$\lng{\self} \geq n$}
  \nomop{Tomar(n, s)}
  \postnl{Retiramos elementos}
  \post{$n = \lng{s\sal} \wedge \self\ent = s\sal \concat \self\sal$}
  \\
  \pre{$\lng{s} \leq \lfloor MAX / 2 \rfloor$}
  \cprenl{Hay sitio en el buffer para dejar la secuencia}
  \cpre{$\lng{\self+s} \leq MAX$}
  \nomop{Poner(s)}
  \postnl{Añadimos una secuencia al buffer}
  \post{$\self\sal = \self\ent \concat s\ent$}
\end{ctadsol}
%
Observad que la especificación contiene también precondiciones para
evitar situaciones de interbloqueo. 

\subsection*{Material a entregar}
El fichero fuente a entregar debe llamarse \url{MultiAlmacenSync.java}.

\subsection*{Material de apoyo}
Se ofrece un esqueleto de código que el alumno debe completar en el fichero
\url{MultiAlmacenSync.todo.java}:
\lstinputlisting{MultiAlmacenSync.todo.java}

El programa principal para probar este ejercicio es el
\url{CC_08_PmultiCSync.java}: 
\lstinputlisting{CC_08_PmultiCSync.java}

\clearpage

\section{Multibuffer con monitores}
%% Aunque mucho folclore insiste en llamar monitores a los métodos
%% synchronized y a los wait sets\footnote{Se recomienda leer el artículo
%%   ``Java's Insecure Parallelism'' de Per Brinch Hansen, el creador de
%%   los monitores.}, lo cierto es que el mecanismo más parecido a
%% monitores del que dispone Java el mecanismo de cerrojos y colas
%% explícitas de espera: \emph{locks \& conditions}
%% (\url{http://docs.oracle.com/javase/tutorial/essential/concurrency/newlocks.html}).

%% \subsection*{Versión con \emph{locks \& conditions}}
%% En este ejercicio, la tarea consiste en resolver el problema anterior
%% usando el mecanismo de cerrojos y colas de espera explícitas
%% (\emph{locks \& conditions}).


%% \subsubsection*{Material a entregar}
%% El fichero fuente a entregar debe llamarse \url{MultiAlmacenLocks.java}.

%% \subsubsection*{Material de apoyo}%
%% Se ofrece un esqueleto de código que el alumno debe completar en el fichero
%% \url{MultiAlmacenLocks.todo.java}:
%% \lstinputlisting{MultiAlmacenLocks.todo.java}

%% El programa principal para probar este ejercicio es el
%% \url{CC_09_PmultiCLocks.java}: 
%% \lstinputlisting{CC_09_PmultiCLocks.java}

%% \subsection*{Versión con los monitores de la CCLib}
%% Lamentablemente, la implementación de \emph{locks \& conditions} de
%% Java no contempla la reentrada al monitor desde un \lstinline{await}
%% de una \lstinline{Condition} (fruto de una ejecución de
%% \lstinline{signal}) como prioritaria sobre las entradas externas al
%% monitor.
En esta versión del ejercicio, la tarea consiste en resolver
el problema anterior usando las clases
\lstinline{es.upm.babel.cclib.Monitor} y
\lstinline{es.upm.babel.cclib.Monitor.Cond}.

\subsubsection*{Material a entregar}
El fichero fuente a entregar debe llamarse \url{MultiAlmacenMon.java}.

\subsubsection*{Material de apoyo}%
Se ofrece un esqueleto de código que el alumno debe completar en el fichero
\url{MultiAlmacenMon.todo.java}:
\lstinputlisting{MultiAlmacenMon.todo.java}

El programa principal para probar este ejercicio es el
\url{CC_09_PmultiCMon.java}: 
\lstinputlisting{CC_09_PmultiCMon.java}

\clearpage
\section{Multibuffer con paso de mensajes}
Se trata de resolver el problema anterior con paso de mensajes usando
la librería
JCSP\footnote{\url{http://www.cs.kent.ac.uk/projects/ofa/jcsp/}}.

\subsubsection*{Material a entregar}
El fichero fuente a entregar debe llamarse \url{MultiAlmacenJCSP.java}.

\subsubsection*{Material de apoyo}%
Se ofrece un esqueleto de código que el alumno debe completar en el fichero
\url{MultiAlmacenJCSP.todo.java}:
\lstinputlisting{MultiAlmacenJCSP.todo.java}

El programa principal para probar este ejercicio es el
\url{CC_10_PmultiJCSP.java}: 
\lstinputlisting{CC_10_PmultiCJCSP.java}

\end{document}

\clearpage
\section*{Especificación ControlAccesoPuente}

\begin{ctadsol}
  \nombrectad{ControlAccesoPuente}
  \\
  \operaciones
  \operacion{Solicitar\_Entrada}{Entrada[e]}
  \operacion{Avisar\_Salida}{Salida[e]}
  \\
  \semdom
  \tipo{ControlAccesoPuente = (sn : \nat \por ns : \nat)}
  \tiponl{Un contador para los coches que van sur-norte y otro para los que van norte-sur. Siempre uno debe ser 0.}
  \donde{
    $Entrada = EN | ES$\\
    $Salida = SN | SS$\\
  }
  \invariante{$(\self.sn > 0 \implies \self.ns = 0)$\\
    $\land (\self.ns > 0 \implies \self.sn = 0)$}
  \\
  \inicial{\self}{$\self.sn = 0 \land \self.ns = 0$}
  \\
  \cprenl{Si se entra por el sur no puede haber coches norte-sur, si se entra por el norte no puede haber coches sur-norte.}
  \cpre{$e = ES \land \self.ns = 0 \lor e = EN \land \self.sn = 0$}
  \nomop{Solicitar\_Entrada(e)}
  \postnl{Incrementamos el número de coches en la dirección adecuada.}
  \post{$e = ES \land \self.sn = \self.sn\ent + 1 \land \self.ns = \self.ns\ent$\\
    $\lor e = EN \land \self.ns = \self.ns\ent + 1 \land \self.sn = \self.sn\ent$}
  \\
  \cprenl{Es imposible que esta operación sea ejecutada por culpa de un coche que no esté cruzando el puente por lo que siempre puede ser ejecutada.}
  \cpre{$\cierto$}
  \nomop{Avisar\_Salida(s)}
  \postnl{Decrementamos el número de coches en la dirección adecuada.}
  \post{$s = SS \land \self.sn = \self.sn\ent \land \self.ns = \self.ns\ent - 1$\\
    $\lor s = SN \land \self.ns = \self.ns\ent \land \self.sn = \self.sn\ent - 1$}
\end{ctadsol}


\end{document}

%%% Local Variables: 
%%% mode: latex
%%% TeX-master: t
%%% TeX-PDF-mode: t
%%% ispell-local-dictionary: "castellano"
%%% End: 
